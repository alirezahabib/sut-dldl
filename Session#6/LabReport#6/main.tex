\documentclass[a4paper,oneside,12pt]{book}

%----------------------------------------------------------------------------------------
%	README!
%   Welcome. It's worth having a read through this file
%   to set up the broad parameters, such as the name of
%   the degree, the school/department, the type of work
%   (dissertation/Final Year Project/report, etc. as well
%   as your own details.
%----------------------------------------------------------------------------------------

%----------------------------------------------------------------------------------------
%	COVER PAGE
%   The cover page is laid out in title/title.tex. You can choose a colour
%   or black and white logo
%----------------------------------------------------------------------------------------

%----------------------------------------------------------------------------------------
%	THESIS INFORMATION
%   Put title, author name, degree, type of work, school, department in here
%   It will be used for the title page and for the embedded PDF information
%----------------------------------------------------------------------------------------

\newcommand{\thesistitle}{
طراحی مدار چاپی\\
واحد محاسبات و منطق
(ALU)
} % Your thesis title, this is used in the title and abstract
\newcommand{\degree}{
\href{http://www.sharif.edu/}{دانشگاه صنعتی شریف}\\
\href{http://ce.sharif.edu/}{دانشکده مهندسی کامپیوتر}
} % Your degree name, this is used in the title page and abstract
\newcommand{\typeofthesis}{Thesis} % dissertation, Final Year Project, report, etc.
\newcommand{\authorname}{
علیرضا حبیب‌زاده
}
\newcommand{\authorid}{\lr{99109393}}

\newcommand{\keywords}{this, that, more} % Keywords for your thesis
\newcommand{\school}{
گزارش کار ششم آزمایشگاه مدارهای منطقی
}

%% Comment out the next line if you don't want a department to appear
\newcommand{\department}{دکتر شاهین حسابی} % Your research group's name and URL, this is used in the title page

\AtBeginDocument{
\hypersetup{pdftitle=\thesistitle} % Set the PDF's title to your title
\hypersetup{pdfauthor=\authorname} % Set the PDF's author to your name
\hypersetup{pdfkeywords=\keywords} % Set the PDF's keywords to your keywords
\hypersetup{pdfsubject=\degree} % Set the PDF's keywords to your keywords
}

%% Language and font encodings
\usepackage[T1]{fontenc} 
\usepackage[utf8]{inputenc}
%\usepackage[english]{babel}
\usepackage{lipsum}
\usepackage{ragged2e} %allows for text alignment preferences

%% Bibliographical stuff
\usepackage[round,sort,comma,numbers]{natbib}

%% Document size
% include showframe as an option if you want to see the boxes
\usepackage[a4paper,top=2.56cm,bottom=2.56cm,left=2.56cm,right=2.56cm, head = 16pt]{geometry}
\setlength{\marginparwidth}{2cm}
%% Useful packages
\usepackage{amsmath}
\usepackage[autostyle=true]{csquotes} % Required to generate language-dependent quotes in the bibliography
\usepackage[pdftex]{graphicx}
\usepackage[colorinlistoftodos]{todonotes}
\usepackage[colorlinks=true, allcolors=black]{hyperref}
\usepackage{xcolor}
\usepackage{pdfpages}
\usepackage{caption} % if no caption, no colon
%\usepackage{sfmath} %use sans-serif in the maths sections too
\usepackage[parfill]{parskip}    % Begin paragraphs with an empty line rather than an indent
\usepackage{setspace} % to permit one-and-a-half or double spacing
\usepackage{enumerate} % fancy enumerations like (i) (ii) or (a) (b) and suchlike
\usepackage{booktabs} % To thicken table lines
\usepackage{fancyhdr}

%\pagestyle{plain} % Embrace simplicity!

%% Uncomment the following block if you want your name and ID at the top of
%% (almost) every page.

\pagestyle{fancy}
\fancyhf{} % sets both header and footer to nothing
\renewcommand{\headrulewidth}{0pt}
\cfoot{\thepage}
%\ifdefined\authorid
%\chead{\it \authorname\ (\authorid)}
%\else
%\chead{\it \authorname}
%\fi
%% End of block

%% It is not a requirement of the university that the font should be sans-serif, but
%% the Mechanical engineers require it. Comment out the following line to disable it
%%\renewcommand{\familydefault}{\sfdefault} %use the sans-serif font as default

%% If you're not using sans-serif, consider using Palatino instead of the LaTeX standard
\usepackage{mathpazo} % Use the Palatino font by default if you prefer it to Computer Modern

\renewcommand{\theequation}{\arabic{equation}} %% use continuous equation numbers

%% Format Chapter headings appropriately
\usepackage{titlesec}
\usepackage{xepersian}
\definecolor{tcdblue}{cmyk}{0.94, 0.38, 0, 0.27}
\newcommand{\hsp}{\hspace{20pt}}
\titleformat{\chapter}[hang]{\Huge\bfseries}{\thechapter\hsp\textcolor{tcdblue}{|}\hsp}{0pt}{\Huge\bfseries}

\title{\thesistitle}
\author{\authorname}

\settextfont{FreeFarsi}
\frontmatter
\begin{document}
\begin{titlepage}

\center % Center everything on the page

%% All the text parameters should be taken from the start of the main.tex file.
%% You should only alter stuff here if you want to change the layout

%----------------------------------------------------------------------------------------
%	LOGO SECTION
%----------------------------------------------------------------------------------------
%% Choose one of the following -- a colour or black-and-white logo

\includegraphics[width=0.3\textwidth]{title/logo.png}\\[1cm] 
%\includegraphics[width=12cm]{title/black-stacked-trinity.jpg}\\[1cm] 
\ifdefined\school
\Large \textsc{گزارش‌کار اول آزمایشگاه مدارهای منطقی} \\[1.5cm] % Minor heading such as course title
\ifdefined\department
\large گزارش‌کار اول آزمایشگاه مدارهای منطقی\\[1.5cm] % Minor heading such as course title
\fi

%----------------------------------------------------------------------------------------
%	TITLE SECTION
%----------------------------------------------------------------------------------------
\makeatletter
\textsc{{ \huge \bfseries آشنایی با محیط‌های شبیه‌سازی}}\\[cm] % Title of your document
 

%----------------------------------------------------------------------------------------
%	AUTHOR SECTION
%----------------------------------------------------------------------------------------

\ifdefined\authorid
علیرضا حبیب‌زاده\\ % Your name
\authorid\\[2cm] % Your Student ID
\else
\textsc{
دکتر شاهین حسابی
}\\[2cm]
\fi

%----------------------------------------------------------------------------------------
%	DATE SECTION
%----------------------------------------------------------------------------------------

\textsc{{\large آبان 1400}}\\[2cm] % Date, change the \today to a set date if you want to be precise

\textcolor{black}{
نویسنده:
علیرضا حبیب‌زاده
\\
شماره دانشجویی:
99109393
}
%----------------------------------------------------------------------------------------
%	TYPE OF THESIS SECTION
%----------------------------------------------------------------------------------------
\vfill

\textsc{\normalsize 
دانشگاه صنعتی شریف\\
دانشکده مهندسی کامپیوتر
}

\vfill % Fill the rest of the page with whitespace

\end{titlepage}
\pagenumbering{roman}
\doublespacing

\newpage
\chapter{مقدمه}
در این جلسه
به کمک نرم‌افزار پروتئوس مدار
ALU
را که در آزمایش قبل ساختیم، بر روی یک برد مدار چاپی
پیاده می‌کنیم.
 
 همه‌ی بخش‌های این آزمایش در نرم‌افزار
 Proteus
 انجام خواهد شد.
 البته از قسمت جدید از نرم‌افزار کمک خواهیم گرفت که مخصوص طراحی مدار چاپی است.

\newpage \tableofcontents

\mainmatter
\chapter{
طراحی
مدار چاپی
یا
PCB
}

\section{
تغییرات مورد نیاز در بخش شماتیک
}
با این که هدف ما در این جلسه تنها طراحی همان شماتیک روی مدار چاپی است، اما باید تغییرات کوچکی در قسمت شماتیک نیز ایجاد کنیم تا بتوانیم مدار را به راحتی و بدون خطا در قسمت طراحی مدار چاپی پیاده کنیم.

\subsection{
ایجاد کانکتور برای ورودی‌ها و خروجی‌های مدار
}
در قسمت شماتیک از قطعات مخصوص دیباگ برای تعیین ورودی‌ها و مشاهده‌ی خروجی‌ها استفاده می‌کردیم.
اما برای طراحی مدار چاپی بهتر است در قسمت شماتیک قطعات کانکتور را هم به صورت موازی قرار دهیم.

قطعات دیباگ در قسمت چاپی نادیده گرفته خواهند شد. (در صورتی که تنظیمات پیش‌فرض این نبود می‌توان در پنجره‌ی تنظیمات هر قطعه تیک
\lr{Exclude from PCB Layout}
آن را فعال کرد)

\subsection{
جایگزین قطعات بدون مدل چاپی
}
برخی از قطعات شماتیک مدل فیزیکی ندارند
در مدار ما گیت‌های استفاده شده گیت‌های مدل پروتئوس هستند و شماتیک یک
IC
فیزیکی نیستند.
باید این قطعات را با جایگزین
مثلا
74xx
خود تعویض کرد.

البته باید دقت کنیم نمی‌توان به سادگی آن‌ها را روی قطعه‌ی قبلی قرار داد چرا که متاسفانه پروتئوس برای جایگزین کردن دو قطعه هر بار از یک
IC
جدید استفاده می‌کند و مثلا گیت اند را همیشه به اولین گیت اند
IC
مقصد مپ می‌کند.
اما ما می‌خواهیم از هر ۴ گیتی که به طور معمول در یک
IC
وجود دارد استفاده کنیم.
برای این کار چاره‌ای جز حذف قبلی‌ها و افزودن دستی جدیدها نداریم.

\subsection{
تغییر مدل چاپی برخی قطعات
}
برخی قطعات مدل‌های چاپی متعددی دارند. در صورتی که برای قطعه‌ای که مدل چاپی داشت با خطایی در قسمت چاپی مواجه شدیم،
(که پیدا کردن قطعات دارای مشکل از لاگ خطاها قابل انجام است.)
کافی است با رفتن به
Properties
آن، مدل چاپی آن را تغییر دهیم.
در تصویر
\eqref{fig:prot1}
نمونه‌ای از این کار آمده.

\begin{figure}[h!]
    \centering
    \includegraphics[width=0.6\textwidth]{images/ALU-PCB-Model.png}
    \caption{
    تغییر مدل چاپی
    }
    \label{fig:prot1}
\end{figure}

\paragraph{}
با انجام این کارها مدار برای طراحی چاپی آماده شده و باید بتوانیم قسمت طراحی چاپی نرم‌افزار را بدون
مشاهده‌ی هیچ خطایی در قسمت لاگ خطاها باز کنیم.
تصویر مدار نهایی اصلاح شده در
\eqref{fig:circuit1}
آمده است.

\begin{figure}[h!]
    \centering
    \includegraphics[width=\textwidth]{images/ALU-PCB.png}
    \caption{
    مدار شماتیک اصلاح شده
    }
    \label{fig:circuit1}
\end{figure}

\newpage

\section{
طراحی
}
پس از رفع خطاها در بخش قبل حال با موفقیت وارد بخش طراحی مدار چاپی نرم‌افزار می‌شویم.

در ابتدای کار با ابزار زرد نوار سمت چپ مرز مدار را مشخص می‌کنیم. سپس می‌توانیم با ابزارهای نوار بالایی زمینه‌ی مدار چاپی را برای تغذیه‌ی قطعات پر کنیم.
نرم‌افزار به طور خودکار وقتی نیاز باشد قسمتی از سمتی که برای تغذیه پر شده را خالی می‌کند و برای سیم‌کشی جدا می‌کند.

در نوار بالا دو گزینه‌ی
\lr{Auto-placer}
و
\lr{Auto-wire}
وجود دارند که عملیات‌های قرار دادن قطعات و 
سیم‌کشی بین آن‌ها را به طور خودکار بر اساس
تنظیماتی که در
\lr{Design Rule Manager}
\eqref{fig:rule}
انجام داده‌اید انجام می‌دهد.
تنظیمات پیش‌فرض نرم‌افزار برای کار ما مناسب هستند.

\begin{figure}[h!]
    \centering
    \includegraphics[width=0.7\textwidth]{images/ALU-PCB-Rule.png}
    \caption{
    قوانین عملیات‌های خودکار
    }
    \label{fig:rule}
\end{figure}

پس از این که از امکان قرارگیری خودکار استفاده کردیم، متوجه می‌شویم که خیلی جالب و مرتب قطعات را نمی‌چیند.
حال به صورت دستی قطعات را می‌چینیم. تا جای ممکن با توجه به اتصالاتی که نرم‌افزار نشان می‌دهد سعی می‌کنیم تا سیم‌کشی ما کمینه شود و از پیچیدگی آن کاسته شود.

در انتها پس از قرار گرفتن قطعات از امکان سیم‌کشی خودکار نرم‌افزار استفاده می‌کنیم.
\eqref{fig:wire}
تنظیمات پیش‌فرض نرم‌افزار مناسب هستند ولی در صورت داشتن سیستم‌قوی می‌توان کمی دفعات بررسی و پاک‌سازی را بیشتر کرد.

\begin{figure}[h!]
    \centering
    \includegraphics[width=\textwidth]{images/ALU-PCB-AutoConnection.png}
    \caption{
    تنظیمات سیم‌کشی خودکار
    }
    \label{fig:wire}
\end{figure}

در انتها مدار ما به شکل
\eqref{fig:final}
در خواهد آمد.

\begin{figure}[h!]
    \centering
    \includegraphics[width=\textwidth]{images/ALU-PCB-PR.png}
    \caption{
    مدار نهایی در محیط نرم‌افزار
    }
    \label{fig:final}
\end{figure}

\begin{figure}[h!]
    \centering
    \includegraphics[width=0.3\textwidth]{images/ALU-PCB-AutoPlacer.png}
    \caption{
    عملیات‌های خودکار در نوار بالا
    }
\end{figure}

\section{
خروجی‌ها و رندر سه‌بعدی
}
نرم‌افزار پروتئوس امکان خروجی طراحی را با فرمت‌های متعددی در اختیار ما می‌گذارد. حتی می‌توانیم یک رندر سه‌بعدی از شکل برد چاپی را
ببینیم و با فرمت‌های معروف شکل سه‌بعدی خروجی بگیریم.

در اینجا به تصاویری از رندر سه‌بعدی و دو تصویر از شکل بالا و پایین برد چاپی اکتفا شده اما در فایل‌های ارسالی خروجی‌های سه‌بعدی نیز قرار داده شده‌اند.

\begin{figure}[h!]
    \centering
    \includegraphics[width=\textwidth]{images/ALU-PCB-3D.png}
    \caption{
    رندر سه‌بعدی برد
    }
\end{figure}

\begin{figure}[h!]
    \centering
    \includegraphics[width=\textwidth]{images/ALU-PCB-BOT.png}
    \caption{
    نمای زیر برد چاپی
    }
\end{figure}

\begin{figure}[h!]
    \centering
    \includegraphics[width=\textwidth]{images/ALU-PCB-TOP.png}
    \caption{
    نمای بالای برد چاپی
    }
\end{figure}



\end{document}